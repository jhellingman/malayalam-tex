% dng.tex -- guide for using Devanagari with Malayalam-TeX
% Copyright 1993 Jeroen Hellingman
% Last edit: 07-JAN-1993

\input dngmacs
\input mmtrmacs
\input dntrmacs

\beginsection {\twelvebf How To Type Devanagari}

Since a package for typesetting Devanagari,
{\tt devnag}, designed by Frans Velthuis, was already available
I~decided to adapt it to Malayalam-\TeX s input conventions.

To start using Devanagari, all you have to do is say

\medskip
{\tt \bslash input dngmacs}

{\tt \bslash input dntrmacs}
\medskip

Somewhere near the start of your document. Then you can switch to
any of the principle languages that use Devanagari-script by saying
{\tt<{}sanskrit>}, {\tt<{}hindi>}, {\tt<{}marathi>}, or {\tt<{}nepali>}.
You will then enter the mode of the indicated language --~together
called {\it Devanagari-mode}.
You can switch back to normal mode by saying respectively
{\tt<{}/sanskrit>}, {\tt<{}/hindi>}, {\tt<{}/marathi>}, or {\tt<{}/nepali>}.
If you wish to use Roman transcription for any of those
languages, you will have to append {\tt .transcription} to the language
name in the start-tag; so Hindi in transcription can be started
with {\tt<{}hindi.transcription>}. No modification of the end-tag is
required (or even allowed).

The following table shows the character(s) you have to type to produce
a Devanagari character.

\bigskip
\input dngtrans
\bigskip

If you want to view or print your document after typing (a part of) it,
you will have to do some extra pre-processing\footnote*{I admit, if you
use several languages in one document, this can get pretty boring.}.
Saying

\medskip
{\tt patc -p dng.pat {\it input-file}.dng {\it temporary-file}.dn }

{\tt devnag {\it temporary-file}.dn {\it output-file}.tex}
\medskip

{\it after} pre-processing for Malayalam, if it is used,
will do all the necessary pre-processing for Devanagari. As you
can see, you will still have to use Frans Velthuis'
pre-processor\footnote{**}{As a matter of fact, I have only changed
the transcription --~all functionality of {\tt devnag}, including
{\tt@}-directives to it is still there.}.
To find out how it should be used exactly, read the manual that
comes with {\tt devnag}.

\endinput
